\documentclass[a4paper,english,12pt]{article}
\usepackage{%
	amsfonts,%
	amsmath,%	
	amssymb,%
	amsthm,%
	babel,%
	bbm,%
	biblatex,%
	caption,%
	centernot,%
	color,%
	enumerate,%
	%enumitem,%
	epsfig,%
	epstopdf,%
	etex,%
	framed,%
	fullpage,%
	geometry,%
	graphicx,%
	hyperref,%
	latexsym,%
	mathptmx,%
	mathtools,%
	multicol,%
	pgf,%
	pgfplots,%
	pgfplotstable,%
	pgfpages,%
	proof,%
	psfrag,%
	subfigure,%	
	tikz,%
	times,%
	ulem,%
	url,%
	xcolor%
}	
\definecolor{shadecolor}{gray}{.95}%{rgb}{1,0,0}
\usepackage[mathscr]{eucal}
\usepgflibrary{shapes}
\usetikzlibrary{%
  arrows,%
  backgrounds,%
  chains,%
  decorations.pathmorphing,% /pgf/decoration/random steps | erste Graphik
  decorations.text,%
  matrix,%
  positioning,% wg. " of "
  fit,%
  patterns,%
  petri,%
  plotmarks,%
  scopes,%
  shadows,%
  shapes.misc,% wg. rounded rectangle
  shapes.arrows,%
  shapes.callouts,%
  shapes%
}

\theoremstyle{plain}
\newtheorem*{thm*}{Theorem}
\newtheorem*{exmp*}{}
\newtheorem{thm}{Theorem}[section]
\newtheorem{lem}[thm]{Lemma}
\newtheorem{prop}[thm]{Proposition}
\newtheorem{cor}[thm]{Corollary}

\theoremstyle{definition}
\newtheorem{defn}[thm]{Definition}
\newtheorem{conj}[thm]{Conjecture}
\newtheorem{exmp}[thm]{Example}
\newtheorem{assum}[thm]{Assumptions}
\newtheorem{axiom}[thm]{Axiom}

\theoremstyle{remark}
\newtheorem{rem}[thm]{Remark}
\newtheorem{note}[thm]{Note}
\newtheorem{con}[thm]{Construction}

\newcommand{\eq}[1]{\begin{align*}#1\end{align*}}
\newcommand{\eqn}[1]{\begin{align}#1\end{align}}
\newcommand{\EQ}[1]{\begin{equation*}#1\end{equation*}}
\newcommand{\EQN}[1]{\begin{equation}#1\end{equation}}
\newcommand{\meq}[2]{\begin{xalignat*}{#1}{#2}\end{xalignat*}}
\newcommand{\meqn}[2]{\begin{xalignat}{#1}{#2}\end{xalignat}}
\newcommand{\norm}[1]{\left\lVert#1\right\rVert}
\newcommand{\indep}{\!\perp\!\!\!\perp}
\DeclarePairedDelimiter\abs{\lvert}{\rvert}%
%\DeclarePairedDelimiter\norm{\lVert}{\rVert}%
\newcommand{\tr}{\operatorname{tr}}
\newcommand{\Var}{\operatorname{Var}}
\newcommand{\Cov}{\operatorname{Cov}}

\newcommand{\D}{\mathbb{D}}
\newcommand{\E}{\mathbb{E}}
\newcommand{\N}{\mathbb{N}}
\newcommand{\Q}{\mathbb{Q}}
\renewcommand{\P}{\mathbb{P}}
\newcommand{\R}{\mathbb{R}}
\newcommand{\Z}{\mathbb{Z}}

\newcommand{\C}{\mathcal{C}}
\newcommand{\cG}{\mathcal{G}}

\newcommand{\sB}{\mathscr{B}}
\newcommand{\sE}{\mathscr{E}}
\newcommand{\sF}{\mathscr{F}}
\renewcommand{\sL}{\mathscr{L}}
\newcommand{\sS}{\mathscr{S}}
\newcommand{\sT}{\mathscr{T}}


\renewcommand{\le}{\leqslant}
\renewcommand{\ge}{\geqslant}

% Debug
\newcommand{\todo}[1]{\begin{color}{blue}{{\bf~[TODO:~#1]}}\end{color}}


\makeatletter
\def\th@plain{%
  \thm@notefont{}% same as heading font
  \itshape % body font
}
\def\th@definition{%
  \thm@notefont{}% same as heading font
  \normalfont % body font
}
\makeatother
\date{}

%opening
\title{Lecture-04: Support Vector Machines}
\date{Aug 14, 2018}
\author{Abhijith J}


\begin{document}
\maketitle
\section{Support Vector Machines}
The margin is defined as,

\begin{equation}
\rho = \underset{(x,y) \in S} {\mathrm{inf}} \frac{|\langle w,x \rangle + b|}{||w||} = \frac{1}{||w||}.
\end{equation}

The primal problem we need to solve then can be stated as, 

\begin{align}
&\text{min} ~~~\frac{1}{2} || w||^2\\
&\text{s.t} ~~~ y_i ( \langle w,x \rangle + b) \geq 1 ~~~~, \forall i \in [m]
\end{align}

There will be a unique solution to this optimization problem as it is a convex optimization problem with affine constraints.

The Lagrangian for this problem can be written as,

\begin{equation}
 \mathcal{L}(w,b, \alpha) = \frac{1}{2} || w||^2 +  \sum_{i = 1}^m \alpha_i (1 -  y_i ( \langle w,x \rangle + b)).
\end{equation}

Let $(w^*, b^*)$ be the solution to the primal problem. Then using KKT conditions we see that they obey the following equations,

\begin{align*}
\nabla_w \mathcal{L}(w^*, b^*, \alpha)~ &= ~ w^* -  \sum_{i =1} ^m \alpha_i~ y_i ~x_i ~=~ 0, \\
\nabla_b \mathcal{L}(w^*, b^*, \alpha)~ & = ~ - \sum_{i= 1}^m ~ \alpha_i~ y_i =~ 0, \\
\alpha_i(1- y_i(\langle w^*,x_i \rangle + b^* ))& = 0 , ~~~\forall i \in [m].
\end{align*} 

From the first condition we find,

\begin{equation}
w^* =  \sum_{i =1} ^m \alpha_i~ y_i ~x_i.
\end{equation}

We see from this expression that a sample contributes to $w^*$ only if the corresponding $\alpha_i$ is non zero. This motivates us to define support vectors.

\begin{defn}[Support Vectors]
Support vectors are defined as the feature vectors for which the corresponding $\alpha_i$ values are non zero.

\begin{equation}
SV(S) = \{ x_i \in S ~:~ \alpha_i  \neq 0 \} =  \{ x_i \in S ~:~ y_i(\langle w^*,x_i \rangle + b^* ) =  1\}.
\end{equation} 

\end{defn}

From this we see that,

\begin{equation}
w^* = \sum_{x_i \in SV(S)} \alpha_i~ y_i ~x_i.
\end{equation}




\begin{rem}
Support vectors fully define the maximum margin hyperplane.Vectors not lying in $SV(S)$ doesn't affect the definition of this hyperplane
\end{rem}


\begin{rem}
$(w^*, b^*)$ is unique but $SV(S)$ need not be.
\end{rem}

Now we derive the dual problem by evaluating the Lagrangian at $(w^*, b^*)$ using the KKT conditions.

\begin{align}
\mathcal{L}(w^*,b^*, \alpha) &= \frac{1}{2} || w^*||^2 +  \sum_{i = 1}^m \alpha_i (1 -  y_i ( \langle w^*,x \rangle + b^*)), \\
&= \frac{1}{2} || \sum_{i =1} ^m \alpha_i~ y_i ~x_i||^2  ~- ~ \sum_{i = 1}^{m} \alpha_i y_i ( \sum_{j=1}^m\alpha_j y_j \langle x_j, x_i \rangle + b^*) +  \sum_{i =1}^{m} \alpha_i,\\
&= \frac12 || \sum_{i =1}^m \alpha_i y_i x_i ||^2  -  \sum_{i,j =1}^m \alpha_i y_i \alpha_j y_j  \langle x_i, x_j \rangle - \sum_{i = 1}^m \alpha_i y_i b^*  + \sum_{i =1}^{m} \alpha_i,\\
&=  -\frac12 || \sum_{i =1}^m \alpha_i y_i x_i || + \sum_{i = 1}^{m} \alpha_ i.
\end{align}

The dual problem then becomes,

\begin{align}
\text{max} &~~~-\frac12 || \sum_{i =1}^m \alpha_i y_i x_i || + \sum_{i = 1}^{m} \alpha_ i \\
\text{s.t}& ~~~\alpha_i \geq 0  ~~~ \text{and} ~~~  \sum_{i =1}^m \alpha_i y_i = 0, ~~~ \forall i \in [m].
\end{align}

The function we are trying to optimize is concave. We can see this by finding its Hessian. Let,

$$ G(\alpha) = -\frac12 || \sum_{i =1}^m \alpha_i y_i x_i || + \sum_{i = 1}^{m} \alpha_ i. $$

The $(i,j)$th element of its Hessian will read,

$$ [\nabla^2_\alpha G]_{ij} =  - \langle y_i x_i, y_j x_j \rangle. $$

From this wee see that $\nabla^2_\alpha G$ has the form $- XX^T$, such that $x_i y_i$ is the $i^{\text{th}}$ row of $X$. This means that $\nabla^2_\alpha G$ is negative semi-definite which in turn implies that $G(\alpha)$ is concave.

The dual problem is a quadratic program which can be solved exactly to get the optimal $\alpha$.  Then $w^*$ will be equal  to $\sum_{i =  1}^m \alpha_i y_i x_i $ by the KKT conditions

~

\begin{note}
$b^*$ can be found as well. We know that if $i \in SV(S)$ then $y_i(\langle w^*,x_i \rangle + b^* ) =  1$. Since $y_i^2 = 1$, we find that,
\begin{equation}
b^* =  y_i - \langle w^*, x_i \rangle ~~~~\forall ~~ x_i \in SV(S).
\end{equation}
\end{note}

\begin{note}
The hypothesis function can then be computed as,

\begin{equation}
h(x) =  \text{sign}(\langle w^*, x \rangle +  b^*).
\end{equation}
\end{note}

\begin{note}

From the KKT conditions,

\begin{equation}
0 =  \sum_{i = 1}^m \alpha_i y_i b^* =  \sum_{i = 1}^m \alpha_i y_i^2  -  \sum_{i,j = 1}^ m \alpha_i \alpha_j y_i y_j \langle x_i, x_j \rangle.
\end{equation}

The second term evaluates to $|| w||^2$ and $y_i^2 = 1$. This gives $ || w||^2 =  \sum_{i = 1}^m \alpha_i $. This gives an expression for the margin in terms of the dual variables,

\begin{equation}
\rho =  \frac{1}{|| w|| ^2} =  \frac{1}{|| \alpha ||_1}.
\end{equation} 
\end{note}

Now we will look at some results that show us why SVMs work well in practice.

\begin{defn}[Leave One Out error]
Given a sample $S$ of size $m$ and a hypothesis $h_S$ the Leave One Out error is defined as,

\begin{equation}
\hat{R}_{LOO}(h_S) = \frac1m \sum_{i = 1}^m \mathbbm{1}_{\{h_{S \backslash \{x_i\}}(x_i) \neq y_i\}}
\end{equation}

\end{defn}

\begin{lem}

\begin{equation}
\mathbbm{E} \hat{R}_{LOO}(h_S) =  \mathop{\mathbb{E}}_{S ^\prime \sim D^{m-1}} R(h_{S^\prime}) .  
\end{equation}

\end{lem}

{\bf Proof:}

~
Using the linearity of expectation,

\begin{align}
\mathbbm{E} \hat{R}_{LOO}(h_S) &= \frac1m \sum_{i = 1}^m  \mathbbm{E} \mathbbm{1}_{\{h_{S \backslash \{x_i\}}(x_i) \neq y_i\}},\\
&=  \mathop{\mathbb{E}}_{S  \sim D^{m}} \mathbbm{1}_{\{h_{S \backslash \{x_i\}}(x_i) \neq y_i\}}, \\
&=  \mathop{\mathbb{E}}_{S ^\prime \sim D^{m-1}} \mathop{\mathbb{E}}_{x_1 \sim D^{m}} \mathbbm{1}_{\{h_{S \backslash \{x_1\}}(x_1) \neq y_1\}},\\
&= \mathop{\mathbb{E}}_{S ^\prime \sim D^{m-1}} R(h_{S^\prime}).
\end{align}

\begin{flushright}
$\blacksquare$
\end{flushright}


\begin{thm}
Let $|S| =  m+1$, then for an SVM,
\begin{equation}
 \mathop{\mathbb{E}}_{S ^\prime \sim D^{m}} R(h_{S^\prime}) = \mathop{\mathbb{E}}_{S \sim D^{m + 1}} \frac{|SV(S)|}{m+1}
\end{equation}
\end{thm}

{\bf Proof:}
We will show that $\hat{R}_{LOO}(SVM) \leq \frac{|SV(S)|}{m+1} $. Then by taking expectation on both sides we get the desired result by applying the previous lemma.

\begin{equation}
\hat{R}_{LOO}(SVM) = \frac{1}{m+1} \sum_{i = 1}^{m+1} \mathbbm{1}_{\{h_{S \backslash \{x_i\}}(x_i) \neq y_i\}}.
\end{equation}

If $i \not\in SV(S)$, then  $h_{S \backslash \{x_i\}} =  h _S$ and hence $ h_{S \backslash \{x_i\}}(x_i) = y_i$. Then taking the contrapositive we see that if  $h_{S \backslash \{x_i\}}(x_i) \neq y_i$ then $i \in SV(S)$. So,

\begin{equation}
\sum_{i = 1}^{m+1} \mathbbm{1}_{\{h_{S \backslash \{x_i\}}(x_i) \neq y_i\}} \leq |SV(S)|. 
\end{equation}

This implies that,

\begin{equation}
\hat{R}_{LOO}(SVM) = \frac{|SV(S)|}{m+1}. 
\end{equation} 

\begin{flushright}
$\blacksquare$
\end{flushright}

\section*{SVM: non-separable case}

In the non-separable case it would not be possible to draw a hyperplane in $\mathbbm{R}^n$ that perfectly separates the two sets of points. More precisely,

\begin{equation}
\forall_{w,b}~~\exists_{i \in [m]} ~~y_i( \langle w, x_i \rangle + b) \leq 1
\end{equation}

To minimize the number of such points we can try to find a hyperplane that  minimizes the empirical error, $$ \mathop{\min}_{w,b} ~~\sum_{i =1}^{m} \mathbbm{1}_{\{ y_i( \langle w, x_i \rangle + b) \leq 1 \}}. $$

But this optimization problem is NP-hard in the dimension of the space and cannot be solved efficiently. Moreover we would like to work with a smooth function to optimize.

\begin{defn}[Outliers]

The set of outliers $O$ is defined as,
$$ O =  \{ x_i \in S ~ :~~ 1 - \xi_i \leq  y_i( \langle w, x_i \rangle + b) \leq 1  \}$$
\end{defn}

The $\xi_i$ are called slack variables. 

We have conflicting objectives here. For one we need to minimize the total slack due to the outliers,   $ \sum_{i \in O} \xi_i^p~~,~~p \geq 1.$

We should also maximize the margin for non-outliers. These two are conflicting objectives. We define a primal problem by deciding on a trade-off between these two. The primal problem is,

\begin{align}
\text{min} ~~& \frac12 ||w||^2 ~~+ ~~c \sum_{i \in O} \xi_i^p \\
~~& \text{s.t} ~~  y_i( \langle w, x_i \rangle + b) \geq   1 - \xi_i ~~,~~ i \in [m] \\
~~~~~&~~~ \xi_i \geq 0
\end{align}  


The constant $c$ fixes the trade-off between the two objectives. The best value of $c$ for a given dataset can be found using cross validation.


\section{Kernel Methods}

Even if the data is non-separable it maybe possible to make it linearly separable by taking it a higher dimensional space (eg: space of polynomials). This lets us use non linear separators which use non-linear decision boundaries. But we saw that the primary operation we need to construct SVMs are inner-products. So instead of working in a higher dimensional space we stick to the original space and compute the inner products using a kernel function. Kernels will usually be symmetric and positive definite.

More precisely let $\Phi : \sX \rightarrow \mathscr{H}$ be the mapping that takes our feature vectors to a higher dimensional space  $\mathscr{H}$. The kernel is then defined by the inner product in this space,

$$K(x, x^\prime) = \langle \Phi(x), \Phi(x^\prime) \rangle_{\mathscr{H}} $$

So the kernel lets us compute inner-products in a higher dimensional space.

\end{document}